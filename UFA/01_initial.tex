\documentclass{article}
\usepackage{amsmath}
\usepackage[T2A]{fontenc}
\usepackage[utf8]{inputenc}
\usepackage[bulgarian]{babel}
\usepackage[pdfencoding=unicode]{hyperref}
\usepackage{biblatex}
\usepackage[style=german]{csquotes}

\title{}
\author{Мартин }
\date{}

\begin{document}

\maketitle

\section{patition}

A partition on an rela line interval [a,b] is a sequence $x_1,..x_{k}$, such that:
$$a=t_{1} < .. < t_{k} = b$$

\section{variation}
given operator $f: [a,b] -> R$, with partition P, define variation of f on P as:
$$V(f, P) = \sum_{i=1}^{n} |f(t_{i})-f(t_{i-1})|$$

\section{Total Variation}
Considering a function f with values in range $[a,b] \in R$
$$V^{b}_{a}=\sup_p \sum_{i=0}^{n_{p}-1} | f(x_{i+1}) - f(x_i) | = \sup_{p} V(f, P)$$

\section{Bounded variation}
a continuous function f is said to be a of bounded variation(BV) on an interval [a,b] $\subset$ R if the total variation on it if finite:
$$f \in BV([a,b]) <-> V_a^b(f) < \infty$$

\paragraph{}
$$BV([a,b]) := \{ f \in C[a,b] | V^{b}_{a}(f) < \infty \}$$
$$$$


\section{refine}:
	not needed ?

\section{monotone examples}
	not needed ?

\section{Banach space}
\paragraph{}

We have to see that $BV[a,b]$ is a Banch Space

\subsection{Lemma for sup-Norm}
\paragraph{}
if $f:[a,b] -> R$ is of bounded variation, then f is bounded and:

$$|| f ||_{\infty} \leq |f(a)| + V^{b}_{a}f$$

\paragraph{proof}
let $a \leq x \leq b$, then:
$$|f(x)-f(a)| \leq V(f,P) \leq V_a^b(f)$$
$$|f(x)| \leq |f(a)| + V_a^b(f)$$

\subsection{}
We will show that $V_{a}^{b}(f)$ is not quite a norm:
\paragraph{Lemma}
Let $f, g \in BV[a,b]$ and $c \in R$, then:
\begin{enumerate}
  \item $V_a^{b}(f) = 0 <-> f$ is a constant
  \item $V_a^b(cf) = |c|V_{a}^b$
  \item $V_a^{b}(f+g) \leq V_{a}^b(f) + V_{a}^b(g) $
\end{enumerate}

\paragraph{proof}
\begin{enumerate}
  \item $V_a^{b}(f) = 0 <-> f$ is a constant
	
	obvious

	Question: we can't consider f=0 ?
  \item $V_a^b(cf) = |c|V_{a}^b$
	Let P be a partition of [a,b] \\
	$$V(cf, P) \leq |c|V(f,P) \leq |c|V_a^b(f)$$

  \item $V_a^{b}(f+g) \leq V_{a}^b(f) + V_{a}^b(g) $
	
	Let P be a partition \\ 
	$$V(f+g, P) \leq V(f,P) + V(g, P)$$ \\
	Lets supreum both sides \\
	$$V^{b}_a(f+g) \leq V^b_a(f) + V^b_a(g)$$
\end{enumerate}

\section{BV norm}

\paragraph{}
We will show that:
$$||f||_{BV} = f(a) + V_a^b(f)$$
is a norm on BV[a,b]

\section{Theorem}
BV[a,b] is complete under $||f||_{BV}$

\section{Question}
B[a,b] - set of all bounded functions ?

\paragraph{proof}
Let $(f_n)$ is a Cauchy seq in BV[a,b], then it's also a Cauchy seq in B[a,b]. \\

$(f_n)$ converges to $f \in B[a,b]$ \\

Lets show that $f \in BV[a,b]$ and $||f-f_n||_{BV} ->0$ \\

Let P be a partiotion in [a,b] \\

Let $\varepsilon>0$, there exists an N, such that for all m,n > N: \\
$$(f_{n}(a)-f_{m}(a)) + (V(f_{n})-V(f_{m}), P) < \varepsilon$$

$$|f_{n}(a)-f(a)|+V(f_n-f, P)= \lim_{m->\infty} \{ |f_{n}(a)-f_m(a)|+V(f_n-f_m, P) \}$$

$$|f_{n}(a)-f(a)|+V(f_n-f, P) \leq \varepsilon, \forall n \geq N$$

This stands true for every partition P in [a,b], so $f_{n}-f \in BV[a,b]$ \\

$$||f_n-f||_{BV} = |f_n(a)-f(a)|+V_a^b(f_n-f) \leq \varepsilon, \forall n \geq N$$

meaning $(f_n)$ converges to $f \in BV[a,b]$ with respect to the $||.||_{BV}$ norm

\section{BV[a,b] is a Banach Space}

\paragraph{} $BV[a,b]$ is a Banach Space with respect to the $||.||_{BV}$ norm:

Considering previous statement:
$$V^b_a=0 <-> f=const$$
Implies that:
$$||f||_{BV}=0<->f=0$$

let $c \in R$ and $f \in BV[a,b]$
$$||cf||_{BV}=f(ca)+V_a^b(cf)=|c|f(a)+|c|V_a^b(f)=|c|||f||_{BV}$$

$$||f-g||_{BV}=(f-g)(a)+V^b_a(f-g) \leq f(a)-g(a)+V^b_a(f)-V^b_a(g)=||g||_{BV}+||f||_{BV}$$


Considering the completeness property from the previous theorem, BV[a,b] is a Banach Space in regards to $||.||_{BV}$

\paragraph{\LARGE Дали следното твърдение е нужно?}

\section{Theorem}
Let $f \in BV[a,b]$ is a real valued function. Then there exists a monotonically increasing functiong $g:[a,b]->R$ such that both g and $g-f$ are increasing. Consequently, f=g-(g-f) is the difference of two increasing functions.

\paragraph{proof}

For $[a, d] \subseteq [a,b]$ Let $V_c^d(f)$ be the total variation of f on [a,d]. Then it can be verified that for any $t \in [a,b]:$
$$V_a^b(f)=V_a^t(f)+V_t^b(f)$$

$$\sum_{i=a}^{b} |f(x_{i})-f(x_{i-1})| = \sum_{i=a}^{t} |f(x_{i})-f(x_{i-1})| + \sum_{i=t}^{b} |f(x_{i})-f(x_{i-1})| $$

We apply $\sup_P$ function: \\
$$\sup_P \sum_{i=a}^{b} |f(x_{i})-f(x_{i-1})| = \sup_P \sum_{i=a}^{t} |f(x_{i})-f(x_{i-1})| + \sup_P \sum_{i=t}^{b} |f(x_{i})-f(x_{i-1})| $$

$$V_a^b(f)=V_a^t(f) + V_t^b(f)$$

Let $ a \leq t \leq s \leq b$
$$V_a^s(f)-V_a^t(f)=V_t^s(f) \geq |f(s)-f(t)| \geq f(s) - f(t)$$

Then $g(t) := V^t_a(f)$ is monotonically increasing. Then:

$$g(s)-g(t)=V^s_a(f)-V^t_a(f)=V_t^s(f) \geq |f(s)-f(t)| \geq f(s)-f(t)$$

\paragraph{\LARGE Дали ще учим за Банахови Алгебри, дали да го включа?}

\end{document}

